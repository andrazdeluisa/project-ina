\documentclass[9pt,twocolumn,twoside]{pnas-new}
% Use the lineno option to display guide line numbers if required.

\templatetype{pnasresearcharticle}

\title{Template for preparing your research report submission to PNAS using Overleaf}

% Use letters for affiliations, numbers to show equal authorship (if applicable) and to indicate the corresponding author
\author[a]{Jan Kenda}
\author[b]{Andraž De Luisa} 

\affil[a]{University of Ljubljana, Faculty of Computer and Information Science, Ljubljana, Slovenia, jk----@student.uni-lj.si
}
\affil[b]{University of Ljubljana, Faculty of Computer and Information Science, Ljubljana, Slovenia, ad9366@student.uni-lj.si
}

% Please give the surname of the lead author for the running footer
\leadauthor{Lead author last name} 

% Please add here a significance statement to explain the relevance of your work
%\significancestatement{}

% Please include corresponding author, author contribution and author declaration information
\authorcontributions{Please provide details of author contributions here.}
\authordeclaration{Please declare any conflict of interest here.}
\equalauthors{\textsuperscript{1}A.O.(Author One) and A.T. (Author Two) contributed equally to this work (remove if not applicable).}
\correspondingauthor{\textsuperscript{2}To whom correspondence should be addressed. E-mail: author.two\@email.com}

% Keywords are not mandatory, but authors are strongly encouraged to provide them. If provided, please include two to five keywords, separated by the pipe symbol, e.g:
\keywords{communities $|$ robustness $|$ node centrality $|$} 

\begin{abstract}
\end{abstract}

\dates{This manuscript was compiled on \today}
\doi{\url{www.pnas.org/cgi/doi/10.1073/pnas.XXXXXXXXXX}}

\begin{document}

\maketitle
\thispagestyle{firststyle}
\ifthenelse{\boolean{shortarticle}}{\ifthenelse{\boolean{singlecolumn}}{\abscontentformatted}{\abscontent}}{}

% If your first paragraph (i.e. with the \dropcap) contains a list environment (quote, quotation, theorem, definition, enumerate, itemize...), the line after the list may have some extra indentation. If this is the case, add \parshape=0 to the end of the list environment.
\dropcap{S}haring ideas, problems and solutions is a fundamental reason why humans are as advanced as they are today.
This is true in all fields of science: biology, physics, mathematics and potentially even more so in computer science.
With the advancements in technology and specifically the internet in the last couple of decades, 
cooperation between researchers and innovators has never been easier.
This is in large part due to Git and other version control systems, 
which allow both researchers and software developers to interact and collaborate on projects with ease,
in turn leading to faster learning, innovation and development of scientific and technological advancements.

But this network has a big drawback: it is a social network.
And as a social network, it has its linchpins - key members of a society which hold everything together.
Without them the network might fall apart. 
These are the people who introduce others. 
The keys that connect people into a community.
And this is what we are interested in.

If we are able to find a community structure in the GitHub network of collaborators \cite{github-network}, 
we can then try to measure how "well connected" each community is.
Using this measure as a criterion, we can then start looking for linchpins - 
in this case nodes, which if removed, would destabilize the community structure of the network as much as possible.
In the real world social network this means removing individuals that we think contribute most to the development of new ideas and projects.
Not necessarily provide solutions themselves, but by connecting others and creating (or just being the core of) communities, contribute to
further technological advancements the most.

In this paper we analyse the ability of several node centrality measures to highlight these crucial nodes for the robustness of the community structure in social networks.

\section*{Related work}

Node centrality:
- bridgeness \cite{Jensen_2015}
- betweenness \cite{freeman}
- pagerank \cite{pagerank}
- community centrality \cite{Newman_2006}
- shapley values centrality \cite{michalak}, \cite{narahari}

Community robustness:
- conductance \cite{kannan-vempala}
- pairwise similarity \cite{bommarito-katz}
- variation of information in perturbed networks \cite{karrer-levina}

Benchmark networks:
- Girvan-Newman \cite{girvan-newman}
- Lancichinetti (LFR) \cite{lfr}
- Affiliation graph model \cite{yang-leskovec}

\section*{Methods}

- find communities in network (if we don't know them)
- rank nodes with respect to a selected measure
- remove best nodes
- compare new communities with previous


\section*{Results}

\section*{Discussion}

\section*{Conclusion}

% Bibliography
\bibliography{biblio}

\end{document}